\begin{itemize}
\item
  Use \texttt{console.log} to print messages.
\item
  Use dotted notation \texttt{X.Y} to get part \texttt{Y} of object \texttt{X}.
\item
  Basic data types are Booleans, numbers, and character strings.
\item
  Arrays store multiple values in order.
\item
  The special values \texttt{null} and \texttt{undefined} mean `no value' and `does not exist'.
\item
  Define constants with \texttt{const} and variables with \texttt{let}.
\item
  \texttt{typeof} returns the type of a value.
\item
  \texttt{for\ (let\ variable\ of\ collection)\ \{...\}} iterates through the values in an array.
\item
  \texttt{if\ (condition)\ \{...\}\ else\ \{...\}} conditionally executes some code.
\item
  \texttt{false}, 0, the empty string, \texttt{null}, and \texttt{undefined} are false; everything else is true.
\item
  Use back quotes and \texttt{\$\{...\}} to interpolate values into strings.
\item
  An object is a collection of name/value pairs written in \texttt{\{...\}}.
\item
  \texttt{object{[}key{]}} or \texttt{object.key} gets a value from an object.
\item
  Functions are objects that can be assigned to variables, stored in lists, etc.
\item
  \texttt{function\ name(...parameters...)\ \{...body...\}} is the old way to define a function.
\item
  \texttt{name\ =\ (...parameters...)\ =\textgreater{}\ \{...body...\}} is the new way to define a function.
\item
  Use \texttt{return} inside a function body to return a value at any point.
\item
  Use modules to divide code between multiple files for re-use.
\item
  Assign to \texttt{module.exports} to specify what a module exports.
\item
  \texttt{require(...path...)} imports a module.
\item
  Paths beginning with `.' or `/' are imported locally, but paths without `.' or `/' look in the library.
\end{itemize}
